% This is the master file of the folder structure. In order to compile your document, run this file. In most LaTeX editors, the master file can be specified such that the document can also be compiled from the other .tex files (in the docs folder).

% First, the preamble needs to be called. This contains all the 'under the hood' stuff for your document.
% This file contains your LaTeX preamble. A preamble is a part of your document where all required packages and macros can be defined. This needs to be done before the \begin{document} command.

% Documentclass:
% Standard LaTeX classes are: article, book, report, slides, and letter. These cover the basis, but are not best. More advanced users might want to try out the KOMA classes or the memoir class. Optional arguments: 10pt. The font size of the main content is set to 10pt with the option between [].
\documentclass[10pt]{article}

% Geometry:
% The papersize of the document is defined with the geometry package. Here, the size is set to A4 with a4paper. Other possibilities are a5paper, b5paper, letterpaper, legalpaper and executivepaper.
\usepackage[a4paper]{geometry}
\usepackage{multicol}
\usepackage[super]{natbib} 

% AMS math packages:
% Required for proper math display.
\usepackage{amsmath,amsfonts,amsthm}

% Graphicx:
% If you want to include graphics in your document, the graphicx package is required.
\usepackage{graphicx}

% Booktabs:
% The booktabs package is needed for better looking tables. 
\usepackage{booktabs}

% SIunitx:
% The SIunitx package enables the \SI{}{} command. It provides an easy way of working with (SI) units.
\usepackage{siunitx}

% URL:
% Clickable URL's can be made with this package: \url{}.
\usepackage{url}

% Caption:
% For better looking captions. See caption documentation on how to change the format of the captions.
\usepackage{caption}

% Hyperref:
% This package makes all references within your document clickable. By default, these references will become boxed and colored. This is turned back to normal with the \hypersetup command below.
\usepackage{hyperref}
	\hypersetup{colorlinks=false,pdfborder=0 0 0}

% Cleveref:
% This package automatically detects the type of reference (equation, table, etc.) when the \cref{} command is used. It then adds a word in front of the reference, i.e. Fig. in front of a reference to a figure. With the \crefname{}{}{} command, these words may be changed.
\usepackage{cleveref}
	\crefname{equation}{equation}{equations}
	\crefname{figure}{figure}{figures}	
	\crefname{table}{table}{tables}

% The title page is created with the command \maketitle which needs to be placed after the \begin{document} command. To create the titlepage, some entries are needed: the name of the autor is defined by \author{}, the title by the entry \title{} and the date by the command \date{}. Note that the current date is displayed with \today.
\author{ 
Graf, Manuel\\
  \texttt{grafm@cip.ifi.lmu.de}
  \and
  Prummer, Michael\\
  \texttt{prummer@cip.ifi.lmu.de}
}
\title{Biofeedback Cardio Training with Video Games}
\date{\today}


\begin{document}

%\tableofcontents

\maketitle

\textbf{\section{abstract}
The Biofeedback System developed at the German Heart Center focuses on the creation of different games that may help to motivate children in the rehabilitation process after surgery, monitoring their vital parameters and adapting to specific conditions.

This paper describes the design and integration of a multi-player video racing ‘exergame’,  controlled via an ergometer and motion sensor, that utilizes the display of Biofeedback to increase or decrease the players physical activity depending on their current heart rate. A preliminary study with young healthy students showed that an aerobic exercise level can be reached, even during short sessions.
\cref{Goe210}
}

\begin{multicols}{2}

\section{related work}

Lorem ipsum dolor sit amet, consetetur sadipscing elitr, sed diam nonumy eirmod tempor invidunt ut labore et dolore magna aliquyam erat, sed diam voluptua. At vero eos et accusam et justo duo dolores et ea rebum. Stet clita kasd gubergren, no sea takimata sanctus est Lorem ipsum dolor sit amet. Lorem ipsum dolor sit amet, consetetur sadipscing elitr, sed diam nonumy eirmod tempor invidunt ut labore et dolore magna aliquyam erat, sed diam voluptua. At vero eos et accusam et justo duo dolores et ea rebum. Stet clita kasd gubergren, no sea takimata sanctus est Lorem ipsum dolor sit amet.
\cite[p. 12ff ]{Goe2010}

\section{methods}
\section{preliminary study}
\section{results}
\section{following work}


\end{multicols}



% The bibliography is printed with \bibliography{}. With the command \bibliographystyle{} a style is picked.

\bibliographystyle{plain}
\bibliography{refs/refs}

% To close your document, add the \end{document} command. Everything after this command will not be processed.
\end{document}