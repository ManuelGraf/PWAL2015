% This is the master file of the folder structure. In order to compile your document, run this file. In most LaTeX editors, the master file can be specified such that the document can also be compiled from the other .tex files (in the docs folder).

% First, the preamble needs to be called. This contains all the 'under the hood' stuff for your document.
% This file contains your LaTeX preamble. A preamble is a part of your document where all required packages and macros can be defined. This needs to be done before the \begin{document} command.

% Documentclass:
% Standard LaTeX classes are: article, book, report, slides, and letter. These cover the basis, but are not best. More advanced users might want to try out the KOMA classes or the memoir class. Optional arguments: 10pt. The font size of the main content is set to 10pt with the option between [].
\documentclass[10pt]{article}

% Geometry:
% The papersize of the document is defined with the geometry package. Here, the size is set to A4 with a4paper. Other possibilities are a5paper, b5paper, letterpaper, legalpaper and executivepaper.
\usepackage[a4paper]{geometry}
\usepackage{multicol}
\usepackage[super]{natbib} 

% AMS math packages:
% Required for proper math display.
\usepackage{amsmath,amsfonts,amsthm}

% Graphicx:
% If you want to include graphics in your document, the graphicx package is required.
\usepackage{graphicx}

% Booktabs:
% The booktabs package is needed for better looking tables. 
\usepackage{booktabs}

% SIunitx:
% The SIunitx package enables the \SI{}{} command. It provides an easy way of working with (SI) units.
\usepackage{siunitx}

% URL:
% Clickable URL's can be made with this package: \url{}.
\usepackage{url}

% Caption:
% For better looking captions. See caption documentation on how to change the format of the captions.
\usepackage{caption}

% Hyperref:
% This package makes all references within your document clickable. By default, these references will become boxed and colored. This is turned back to normal with the \hypersetup command below.
\usepackage{hyperref}
	\hypersetup{colorlinks=false,pdfborder=0 0 0}

% Cleveref:
% This package automatically detects the type of reference (equation, table, etc.) when the \cref{} command is used. It then adds a word in front of the reference, i.e. Fig. in front of a reference to a figure. With the \crefname{}{}{} command, these words may be changed.
\usepackage{cleveref}
	\crefname{equation}{equation}{equations}
	\crefname{figure}{figure}{figures}	
	\crefname{table}{table}{tables}

% The title page is created with the command \maketitle which needs to be placed after the \begin{document} command. To create the titlepage, some entries are needed: the name of the autor is defined by \author{}, the title by the entry \title{} and the date by the command \date{}. Note that the current date is displayed with \today.
\author{ 
Manuel Graf\\
  \texttt{grafm@cip.ifi.lmu.de}
  \and
  Michael A. Prummer\\
  \texttt{prummer@cip.ifi.lmu.de}
}
\title{Design of a Multiplayer Exergame for a Biofeedback System using Unity 3D
}
\date{\today}


\begin{document}

%\tableofcontents

\maketitle

\textbf{\section{abstract}
The Biofeedback System developed at the German Heart Center focuses on the creation of different games that may help to motivate children in the rehabilitation process after surgery, monitoring their vital parameters and adapting to specific conditions.

This paper describes the design and integration of a multi-player video racing ‘exergame’,  controlled via an ergometer and motion sensor, that utilizes the display of Biofeedback to increase or decrease the players physical activity depending on their current heart rate. A preliminary study with young healthy students showed that an aerobic exercise level can be reached, even during short sessions.
\cref{Goe210}
}

\begin{multicols}{2}

\section{Introduction}

Biofeedback is a relatively new method of treatment for mainly psychosomatic diseases. It can be used in many different ways through visualizing information about the physiological functions of the body. Biofeedback can be applied using various types of data, such as heart rate, brainwaves, body temperature or skin conductance of a patient. This data may help to improve the health and performance of an individual during a physical exercise or a treatment. Visualising the participants vital parameters improves the perception of his own body and thus his health. By adapting to the given feedback patients can learn to affect their own body functions.\cite{BF2007} The recorded biofeedback data of patient during an exercise could be visualized and analysed across multiple sessions. 

Besides this, Biofeedback could also be used to promote and motivate exercise for rehabilitation. Therefore, the biofeedback system can be combined with interactive applications, which hide the exhausting task of exercising behind a video game. Most things could be done more easily if we think of it as a game instead of work. Especially for rehabilitation, it may help to make boring exercises more entertaining, that have to be repeated continuously over a very long period, but are crucial for physical recovery. In order to combine interactive games with
exercise and biofeedback, we will introduce the concept of exergames (EG) and serious games (SG).

Games designed for a specific purpose, teaching a skill or educate can be referred as serious games. \cite{Derryberry} Serious games are a steady growing and relevant market since video games get more accepted in society. \cite{SGIndustry} The game prototype we seek to develop for the biofeedback system at the “Deutsches Herzzentrum” (DHZ), can be classified as a serious game based on the player movement or more specific, an exergame. “The activity of playing video games that involve physical exertion and are thought of as a form of exercise.” %\cite{ExergameDef}

\section{Related Work}
There are many successful commercial “exergame platforms” like the Nintendo Wii or Wii Fit. Studies about the efficiency of the Wii according to the exercise, mostly shows different results. \cite{Baranowski2012} The major problem with most of the consumer systems is that the patient could easily adopt a strategy to trick the system to be successful in the game. Accordingly, we have to use our own motion tracking system with a camera to avoid such behaviour in first place. Thus, exergames are not full replacements for real sports, they can be a great support in improving balance, endurance or help be more aware of specific problems. There are also many scientific evaluated studies from Baranowski \cite{Baranowski2008} and Kretschmann %\cite{Kretschmann2010}, which have shown positive health related behaviour changes in working together with video game setups.

Information we gathered from the evaluation of the previously developed exergames at the DHZ showed that single-player video games lacked the needed methods to sustain motivation. Playing the games for the first time could be a lot of fun while the concept of controlling the game over the biofeedback sensors is new to the player. But as already stated, this setup won't work quite well in keeping up the long term motivation for proceeding sessions.

\section{Game Concept}


\section{Input Devices and Sensors}


\section{Motivational Aspects}

A big challenge in creating a high qualitative exergame is to make it replayable and keep up a steady motivation over a longer period of time. Our goal was to develop a multiplayer racing exergame, which could be played together with up to eight players or an artificial intelligence. The game is meant to be embedded in the biofeedback system at the DHZ. We will also try to improve the game experience through controlling the karts with a motion sensor and through the implementation of various game elements, such as weapons system or collectable items, similar to the classic “Super Mario Kart” %\cite{NitendoWiiMario} combat racing game. The game prototype has to be developed with the Unity3D game engine, based on the experience of the former developed games at the DHZ. The game should provide an initial motivation to start with exercising and support a better long term motivation through the multiplayer extension. Therefore, we what the game to be appropriate for the longer task of rehabilitation, especially with the target group of children and adolescents. 

Stuart Gray researched the connection between physical extent in exergamess and the motivational pull of head-to-head and leader board competitive games \cite{Gray2013}. Firstly, he analyzed different conducted studies which vary in their results regarding the effect of multiplayer competitive games on intrinsic motivation. The analyzed studies led him to the conclusion that competitive gameplay has a positive effect on highly competitive individuals whereas it tends to have negative or no effect at all on the intrinsic motivation of lowly competitive individuals. His main study was conducted on 33 participants, aged from 15 to 65, with an average age of 25, over a period of 4 weeks, and was analyzing the impact of competitive EG play on energy expenditure. Participants who played any competitive game mode spent significantly more time in anaerobic training zones (0.82 * HRmax - 1 * HRmax), suggesting that competitive conditions are more effective for energy expenditure than non-competitive conditions.


\section{Biofeedback Adaptations}

\section{Preliminary Study}

\section{Conclusion and Following Work}


\end{multicols}



% The bibliography is printed with \bibliography{}. With the command \bibliographystyle{} a style is picked.

\bibliographystyle{plainnat}
\bibliography{refs/refs}

% To close your document, add the \end{document} command. Everything after this command will not be processed.
\end{document}