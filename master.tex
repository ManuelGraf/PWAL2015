% This is the master file of the folder structure. In order to compile your document, run this file. In most LaTeX editors, the master file can be specified such that the document can also be compiled from the other .tex files (in the docs folder).

% First, the preamble needs to be called. This contains all the 'under the hood' stuff for your document.
\input{docs/preamble}

% The title page is created with the command \maketitle which needs to be placed after the \begin{document} command. To create the titlepage, some entries are needed: the name of the autor is defined by \author{}, the title by the entry \title{} and the date by the command \date{}. Note that the current date is displayed with \today.
\author{ 
Manuel Graf\\
  \texttt{grafm@cip.ifi.lmu.de}
  \and
  Michael A. Prummer\\
  \texttt{prummer@cip.ifi.lmu.de}
}
\title{Design of a Multiplayer Exergame for a Biofeedback System using Unity 3D
}
\date{\today}


\begin{document}

%\tableofcontents

\maketitle

\textbf{The Biofeedback System developed at the German Heart Center focuses on the creation of different games that may help to motivate children in the rehabilitation process after surgery, monitoring their vital parameters and adapting to specific conditions.

This paper describes the design and integration of a multi-player video racing ‘exergame’,  controlled via an ergometer and motion sensor, that utilizes the display of Biofeedback to increase or decrease the players physical activity depending on their current heart rate. A preliminary study with young healthy students showed that an aerobic exercise level can be reached, even during short sessions.
\cite{Goe2010}
}

\begin{multicols}{2}

\section{Introduction}

Biofeedback is a relatively new method of treatment for mainly psychosomatic diseases. It can be used in many different ways through visualizing information about the physiological functions of the body. Biofeedback can be applied using various types of data, such as heart rate, brainwaves, body temperature or skin conductance of a patient. This data may help to improve the health and performance of an individual during a physical exercise or a treatment. Visualising the participants vital parameters improves the perception of his own body and thus his health. By adapting to the given feedback patients can learn to affect their own body functions.\cite{Goe2010} The recorded biofeedback data of patient during an exercise could be visualized and analysed across multiple sessions. Besides this, Biofeedback could also be used to promote and motivate exercise for rehabilitation. Therefore, the biofeedback system can be combined with interactive applications, which hide the exhausting task of exercising behind a video game. Most things could be done more easily if we think of it as a game instead of work. Especially for rehabilitation, it may help to make boring exercises more entertaining, that have to be repeated continuously over a very long period, but are crucial for physical recovery. In order to combine interactive games with
exercise and biofeedback, we will introduce the concept of exergames and serious games.
Games designed for a specific purpose, teaching a skill or educate can be referred as serious games. [2] Serious games are a steady growing and relevant market since video games get more accepted in society. [3] The game prototype we seek to develop for the biofeedback system at the “Deutsches Herzzentrum” (DHZ), can be classified as a serious game based on the player movement or more specific, an exergame. “The activity of playing video games that involve physical exertion and are thought of as a form of exercise.” [4] 
\section{Related Work}
There are many successful commercial “exergame platforms” like the Nintendo Wii or Wii Fit. Studies about the efficiency of the Wii according to the exercise, mostly shows different results. [5] The major problem with most of the consumer systems is that the patient could easily adopt a strategy to trick the system to be successful in the game. Accordingly, we have to use our own motion tracking system with a camera to avoid such behaviour in first place. Thus, exergames are not full replacements for real sports, they can be a great support in improving balance, endurance or help be more aware of specific problems. There are also many scientific evaluated studies from Baranowski [6] and Kretschmann [7], which have shown positive health related behaviour changes in working together with video game setups. 
\section{Methods}
A big challenge in creating a high qualitative exergame is to make it replayable and keep up a steady motivation over a longer period of time. Our goal was to develop a multiplayer racing exergame, which could be played together with up to eight players and versus an artificial intelligence. The game is meant to be embedded in the biofeedback system at the DHZ. We will also try to improve the game experience through controlling the karts with a motion sensor and through the implementation of various game elements, such as weapons system or collectable items, similar to the classic “Super Mario Kart” [8] combat racing game. The game prototype has to be developed with the Unity3D game engine, based on the experience of the former developed games at the DHZ. The game should provide an initial motivation to start with exercising and support a better long term motivation through the multiplayer extension. Therefore, we what the game to be appropriate for the longer task of rehabilitation, especially with the target group of children and adolescents. There are different important points we learned from the evaluation of the previously developed
exergames at the DHZ. In one of the games, a car have to be controlled by a gamepad, sitting on an ergometer. The goal is to collect as many coins as possible while driving along a road. Playing the game at first times could be a lot of fun while the concept of controlling the game over the biofeedback sensors is new to the player. But as said before, this setup won't work quite well in keeping up the long term motivation for proceeding sessions, how it will be needed for rehabilitation.\cite[p. 12ff ]{Goe2010}


\section{preliminary study}
\section{results}
\section{following work}


\end{multicols}



% The bibliography is printed with \bibliography{}. With the command \bibliographystyle{} a style is picked.

\bibliographystyle{plain}
\bibliography{refs/refs}

% To close your document, add the \end{document} command. Everything after this command will not be processed.
\end{document}